\documentclass[10pt,a4paper]{article}
\usepackage[utf8]{inputenc}
\usepackage[french]{babel}
\usepackage[T1]{fontenc}
\usepackage{amsmath}
\usepackage{amsfonts}
\usepackage{amssymb}
\usepackage{hyperref}
\begin{document}
\title{Compilateur pour Petit Julia}
\author{Hector BUFFIÈRE et Thomas LAURE}
\date{1er semestre 2020-2021}
\maketitle

\section{Introduction}
Projet réalisé dans le cadre du cours Langages de Programmation et Compilation du 1er semestre de L3 à l'Ecole Normale Supérieure.

Sujet : \url{https://www.lri.fr/~filliatr/ens/compil/projet/sujet-v2.pdf}

Ce projet a été réalisé dans le langage Ocaml.

\section{Analyse lexicale et syntaxique}
\subsection{Analyse lexicale}
L'analyse lexicale a été réalisée avec Ocamllex.

Le point-virgule automatique en fin de ligne est traité à l'aide d'une référence booléenne globale. Quand on rencontre un lexème, on met celle-ci à jour selon si ce lexème serait suivi d'un point-virgule automatique si l'on rencontrait un retour chariot ensuite. Quand on rencontre un retour chariot, on renvoie le lexème ";" si la valeur de la référence était Vrai. 

Petit Julia accepte que des entiers soient suivis directement d'une parenthèse ou d'une variable, on renvoie donc des lexèmes dédiés dans ce cas, et de même pour les variables suivies ou suivant directement une parenthèse.

Enfin, Ocaml n'acceptant pas les entiers au delà de $2^{62}$, on utilise le module $Int64$ d'Ocaml pour représenter les entiers 64 bits de Petit Julia.

Nous avons également ajouté les lexèmes / et $\setminus $ qui n'étaient pas demandés par le sujet.

\subsection{Arbre de syntaxe abstraite}
Les expressions sont représentées par un type construit contenant leur localisation dans le fichier, et la nature du nœud de l'arbre qu'elles représentent.

Les déclarations de fonctions et de structures sont des types construits constitués de leurs noms, de leurs arguments, de leurs localisations dans le fichier, de leurs instructions pour les premières et de leur caractère mutable ou non pour les secondes.

Les paramètres sont aussi un type construits donnant comme information leur nom, leur type et leur localisation dans le fichier.

\subsection{Analyse syntaxique}
L'analyse syntaxique a été réalisée avec Menhir.

L'analyseur tient compte du sucre syntaxique proposé. La fonction div est remplacée par l'opérateur binaire $/$, la fonction $println$ par la fonction print à laquelle est ajoutée l'expression $\setminus n$ après les arguments. Les types omis dans les déclarations de paramètres sont bien évalués à $Any$.

Les éléments des blocs (listes d'expressions) et des listes de paramètres peuvent être séparés par n'importe quel nombre de point-virgule, et la grammaire en tient compte.

Les précédences et associativités sont celles indiquées dans le sujet. Un conflit lié au return a été résolu en déclarant les lexèmes centraux des expressions comme de précédence plus grande que celle du return. Les conflits liés à la succession d'une expression et d'un bloc sans séparateur dans les if, while et for ont été résolu en séparant la production (mot-clé expression) de la production ((mot-clé expression) instructions fin) et en associant une précédence plus grande à la première.

\section{Typage}
\end{document}